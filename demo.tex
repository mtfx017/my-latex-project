\documentclass[12pt, a4paper, oneside]{ctexart}
\usepackage{amsmath, xfrac, amsthm, amssymb, graphicx}
\usepackage[bookmarks=true, bookmarksnumbered=true, colorlinks, citecolor=blue, linkcolor=black]{hyperref}
\usepackage{indentfirst}
\usepackage{geometry}
\usepackage{fancyhdr}
\usepackage{titlesec}
\usepackage{enumitem}
\usepackage{setspace}
\usepackage{lipsum}

% 页边距设置
\geometry{a4paper, left=2.5cm, right=2.5cm, top=2.5cm, bottom=2.5cm}

% 页眉页脚设置
\pagestyle{fancy}
\fancyhf{}
\fancyhead[L]{我的第一个\LaTeX 文档}
\fancyhead[R]{\thepage}
\fancyfoot[C]{\today}

% 章节标题格式
\titleformat{\section}
  {\normalfont\Large\bfseries}{\thesection}{1em}{}
\titleformat{\subsection}
  {\normalfont\large\bfseries}{\thesubsection}{1em}{}

% 段落间距设置
\setlength{\parskip}{0.5em}
\setlength{\parindent}{2em}

% 标题页设置
\title{\Huge\bfseries 我的第一个\LaTeX 文档}
\author{\Large 我喜欢满天繁星.}
\date{\Large \today}

\begin{document}

% 封面页
\begin{titlepage}
    \centering
    \vspace*{5cm}
    {\Huge\bfseries 我的第一个\LaTeX 文档\par}
    \vspace{1.5cm}
    {\Large 我喜欢满天繁星.\par}
    \vfill
    {\large \today\par}
\end{titlepage}

\thispagestyle{empty}

\newpage

\thispagestyle{empty}

\tableofcontents

\newpage

\setcounter{page}{1}

\section{plzover}

\subsection{导数}

\begin{enumerate}[leftmargin=*]
    \item 已知函数 $f(x)=e^x + a \sin \left( x + \dfrac{\pi}{4} \right) - a^2 (x+1)$.
          \begin{enumerate}
              \item $g(x)=f(x) - a \sin \left( x + \dfrac{\pi}{4} \right)$, 试讨论 $g(x)$ 的单调性.
              \item $f(x) \geq 0$ 对 $\forall x \in (-1, +\infty)$ 均成立, 求 $a$ 的取值范围.
          \end{enumerate}
          \vspace*{2\baselineskip}
    \item 已知函数$f(x)=e^{x-a} - a \ln x + 1$, $a \in (-1, 0)$.
    
    证明:存在唯一的 $x_0$ 使得 $f(x_0) = 0$, 且 $x_0$ 随 $a$ 递减.
          \vspace*{2\baselineskip}
    \item 已知函数$f(x)=x^a\ln x$, $a\in \mathbb{R} $.
          \begin{enumerate}
              \item 讨论$f(x)$的单调性.
              \item 当$x\geqslant 1$, $a\geqslant 1$时, 证明:

                    $f(x)\geqslant x^{a-1}(x-1)$以及$f(x)\geqslant -\dfrac{1}{2}(x-1)(x-3)[1+a(x-1)]$.
              \item 当$a\geqslant 1$, 证明:

                    $\dfrac{n(n-1)}{2}\leqslant \sum_{i=1}^{n} i^a \ln i \leqslant \dfrac{[a(n-1)+(n-2)]}{(a+1)^2}n^{a+1} +\dfrac{1}{(a+1)^2}$, $i\in \mathbb{N} _+$.
          \end{enumerate}
          \vspace*{2\baselineskip}
    \item 已知函数$f(x)=\ln x-3ax^2+2$.
          \begin{enumerate}
              \item $f(x)$有两个零点, 求$a$的取值范围.
              \item 若函数 $f(x)$ 有两个不同的零点$x_1, x_2, (x_1<x_2)$, 证明: $x_1+x_2>\sqrt{\dfrac{2}{3a}} $.
          \end{enumerate}
\end{enumerate}

\subsection{解三角形}

\begin{enumerate}[leftmargin=*]
      \item 在$\triangle ABC$中, $\overrightarrow{CP}=\dfrac{\sqrt{10} }{2}\overrightarrow{CA}+\overrightarrow{CB}   $, $\cos \angle ACB=\dfrac{\sqrt{10} }{10}$, $\left\lvert CP\right\rvert  =\sqrt{3} $, 求$a+2b$的最大值.
  \end{enumerate}

\section{我喜欢满天繁星.}

\subsection{导数}

\begin{enumerate}[leftmargin=*]
    \item 已知函数 $f(x)=ax+\dfrac{b}{x}-\left\lvert \ln x\right\rvert$ 有两个极值点 $x_1, x_2$.
    
    证明:
          \begin{enumerate}
              \item $\left\lvert b-a\right\rvert <1$.
              \item $f(x_1) = f(x_2)$ 是 $a=b$ 的充分必要条件.
          \end{enumerate}
          \vspace*{2\baselineskip}
    \item 已知函数 $f(x)=\left( x-1 \right)^2 e^x$, 方程 $f(x)=a$ 有三个根 $x_1, x_2, x_3$ ($x_1 < x_2 < x_3$), 证明:
          \begin{enumerate}
              \item $x_2 + x_3 < 2$.
              \item 当 $0 < a < 1$ 时, $\left\lvert x_2 - x_3 \right\rvert < 2 \sqrt{a}$.
          \end{enumerate}
          \vspace*{2\baselineskip}
    \item 已知函数 $f(x)=e^{ax} + a^2 x^2 + bx$.
          \begin{enumerate}
              \item 若函数 $g(x)=f(x) - a^2 x^2$ 有两个不同的零点, 求 $\dfrac{b}{a}$ 的取值范围.
              \item 设 $x_0$ 是函数 $f(x)$ 的极值点, 证明 $f(x_0) \leqslant 1-a^2 x_0^2$.
          \end{enumerate}
          \vspace*{2\baselineskip}
    \item 已知函数 $f(x)=\left(1-\ln x\right)x^m$, $g(x)=\left(1-x\right)n^x$, $(m, n>0)$.
          \begin{enumerate}
              \item 证明: $f(x)$ 和 $g(x)$ 都一定有零点.
              \item 记 $f(x)$ 和 $g(x)$ 的最大值分别为 $M, N$.当 $M \geqslant 2 \ln 2$ 时, 若 $M \geqslant N$, 求 $n$ 的取值范围.
          \end{enumerate}
\end{enumerate}

\subsection{数列}

\begin{enumerate}[leftmargin=*]
    \item 在数列 $\{a_n\}$ 中, $n a_{n+1} = \left( n + 1 \right)^2 a_n$, $a_1 = 1$.
          \begin{enumerate}
              \item 求数列 $\{a_n\}$ 的通项公式.
              \item 求数列 $\{a_n\}$ 的前 $n$ 项和 $S_n$.
          \end{enumerate}
          \newpage
    \item 在数列 $\{a_n\}$ 中, 若 $S_{n+1} = (n+2)S_n + n+1$, $S_2 = 5$, 求数列 $\{a_n\}$ 的通项公式.
          \vspace*{2\baselineskip}
    \item 在数列 $\{a_n\}$ 中, 若 $\left( \sqrt{S_{n+1}} - \sqrt{S_n} \right) \left( \sqrt{n+1} - \sqrt{n} \right) = 1$, $a_1=1$.
    
    证明: 数列 $\{a_n\}$ 为常数列.
          \vspace*{2\baselineskip}
    \item 在数列 $\{a_n\}$ 中, 若 $\dfrac{S_n}{n^2} + n = \dfrac{a_n}{n^2} + \dfrac{1}{n}$.
          \begin{enumerate}
              \item 求 $\{a_n\}$ 的通项公式.
              \item 令 $b_n = \dfrac{1}{a_n}$, 数列 $\{b_n\}$ 前 $n$ 项和为 $T_n$, 证明:$-\dfrac{1}{3} < T_n \leqslant -\dfrac{1}{6}$.
          \end{enumerate}
\end{enumerate}

\subsection{解析几何}

\begin{enumerate}[leftmargin=*]
    \item 已知椭圆 $\varGamma : \dfrac{x^2}{a^2} + \dfrac{y^2}{b^2} = 1$ $(a > b > 0)$, 过左顶点 $A$ 的直线 $l_1$ 交椭圆 $\varGamma$ 于点 $M$ (异于点 $A$), 过坐标原点 $O$ 的直线 $l_2$ 交椭圆 $\varGamma$ 于点 $N$, 直线 $l_1$ 与直线 $l_2$ 交于点 $P$, $\overrightarrow{MN} = \lambda \overrightarrow{OM}$ $(\lambda \neq 0)$, $\overrightarrow{OM} \cdot \overrightarrow{ON} \in (-4 , 1)$.
          \begin{enumerate}
              \item 求椭圆 $\varGamma$ 的方程.
              \item 是否存在定点 $F_1, F_2$, 使得 $\left\lvert PF_1 \right\rvert - \left\lvert PF_2 \right\rvert$ 为定值.若存在, 请求出 $F_1, F_2$ 的坐标; 若不存在, 请说明理由.
          \end{enumerate}
          \vspace*{2\baselineskip}
    \item 已知椭圆 $\varGamma : \dfrac{x^2}{4} + y^2 = 1$, 点$P\left(m, n\right)$在圆$M: x^2+\left(y+3\right)^2=1$上, $PA$, $PB$是椭圆$\varGamma$的两条切线, $A, B$是切点.
    \begin{enumerate}
        \item 证明: 直线$AB$的方程为$mx+4ny=4$.
        \item 求$S_{\triangle PAB}$的取值范围.
    \end{enumerate}
\end{enumerate}

\subsection{创新题}

\begin{enumerate}[leftmargin=*]
    \item 已知点$A\left(e^x, x\right)$, 点$B\left(x, \ln x\right)$, 点$C\left(\ln x,x\right)$.
          \begin{enumerate}
              \item 请探究$A,B,C$三点能否共线.如能, 请说明理由;如不能, 也请说明理由.
              \item 证明: $S_{\triangle ABC}>1$.
              \item 证明: $\sqrt{2}\left\lvert AB\right\rvert \geq \left\lvert AC\right\rvert  $.
          \end{enumerate}
\end{enumerate}
\end{document}
